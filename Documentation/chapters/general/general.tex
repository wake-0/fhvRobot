\section{Allgemein}
In diesem Kapitel werden allgemeine Aspekte zum Robot Projekt erläutert. Dazu zählen insbesondere die durch das Studienprojekt definierten Anforderungen, zusätzlich durch die Studierenden gesetzte Anforderungen und die erreichten Ergebnisse hinsichtlich dieser Anforderungen.

\subsection{Vorgegebene Anforderungen an das Projekt}
Eine Auflistung aller vorgegebenen Anforderungen, insbesondere funktionale Anforderungen an das Betriebssystem sind in Tabelle \ref{table:Prescribed-Requirements} angegeben.

\begin{table}[H]
\begin{tabular}{ p{5cm}| p{9cm} }
  \textbf{Anforderung} & \textbf{Erklärung} \\ 
  \hline
  XX & xxx. \\
  XX & xxx. \\
 \end{tabular}
 \caption{Vorgegebene Anforderungen}
 \label{table:Prescribed-Requirements}
\end{table}

\subsection{Eigene Anforderungen an das Projekt}
Zusätzlich zu den oben angeführten Anforderungen wurden weitere, nicht-funktionale Anforderungen an das Robot Projekt, durch die an der Entwicklung beteiligten Studierenden, definiert. Eine Auflistung aller eigenen Anforderungen sind in Tabelle \ref{table:Own-Requirements} angegeben.

\begin{table}[H]
\begin{tabular}{ p{5cm}| p{9cm} }
  \textbf{Anforderung} & \textbf{Erklärung} \\ 
  \hline
  Hoher Abstraktionsgrad & Alle Komponenten des Projekts sollen einen hohen Abstraktionsgrad aufweisen. \\
  Intuitiver Aufbau & Die Komponenten des Projekts sollen eine intuitive Programmierschnittstelle aufweisen. \\
  Leichte Erweiterbarkeit & Mögliche Erweiterungen sollen ohne große Veränderungen an der Architektur umgesetzt werden können. \\
  Einfache Wartung & Das Robot Projekt soll eine einfache Wartbarkeit hinsichtlich Fehlern aufweisen. \\
 \end{tabular}
 \caption{Vorgegebene Anforderungen}
 \label{table:Own-Requirements}
\end{table}

\subsection{Erfüllung der Anforderungen}
Im Allgemeinen wurden alle zuvor erwähnten funktionalen Anforderungen an das Projekt erfüllt. Einzelne Verbesserungs- bzw. Erweiterungsmöglichkeiten können aus Kapitel \ref{summary} werden. 
\pagebreak 